% !TEX encoding = UTF-8 Unicode

% This is a simple template for a LaTeX document using the "article" class.
% See "book", "report", "letter" for other types of document.

\documentclass[11pt]{article} % use larger type; default would be 10pt

\usepackage[utf8]{inputenc} % set input encoding (not needed with XeLaTeX)
\usepackage{url}
\usepackage{color}
%%% Examples of Article customizations
% These packages are optional, depending whether you want the features they provide.
% See the LaTeX Companion or other references for full information.

%%% PAGE DIMENSIONS
\usepackage{geometry} % to change the page dimensions
\geometry{a4paper} % or letterpaper (US) or a5paper or....
% \geometry{margin=2in} % for example, change the margins to 2 inches all round
% \geometry{landscape} % set up the page for landscape
%   read geometry.pdf for detailed page layout information

\usepackage{graphicx} % support the \includegraphics command and options

% \usepackage[parfill]{parskip} % Activate to begin paragraphs with an empty line rather than an indent

%%% PACKAGES
\usepackage{booktabs} % for much better looking tables
\usepackage{array} % for better arrays (eg matrices) in maths
\usepackage{paralist} % very flexible & customisable lists (eg. enumerate/itemize, etc.)
\usepackage{verbatim} % adds environment for commenting out blocks of text & for better verbatim
\usepackage{subfig} % make it possible to include more than one captioned figure/table in a single float
% These packages are all incorporated in the memoir class to one degree or another...
\usepackage{url}

%%% HEADERS & FOOTERS
\usepackage{fancyhdr} % This should be set AFTER setting up the page geometry
\pagestyle{fancy} % options: empty , plain , fancy
\renewcommand{\headrulewidth}{0pt} % customise the layout...
\lhead{}\chead{}\rhead{}
\lfoot{}\cfoot{\thepage}\rfoot{}

\long\def\omitit #1{}

%%% SECTION TITLE APPEARANCE
\usepackage{sectsty}
\allsectionsfont{\sffamily\mdseries\upshape} % (See the fntguide.pdf for font help)
% (This matches ConTeXt defaults)

%%% ToC (table of contents) APPEARANCE
\usepackage[nottoc,notlof,notlot]{tocbibind} % Put the bibliography in the ToC
\usepackage[titles,subfigure]{tocloft} % Alter the style of the Table of Contents
\renewcommand{\cftsecfont}{\rmfamily\mdseries\upshape}
\renewcommand{\cftsecpagefont}{\rmfamily\mdseries\upshape} % No bold!

%%% END Article customizations

%%% The "real" document content comes below...


\title{\textbf{CMPSC 300 Introduction to Bioinformatics\\Syllabus}}
\author{Fall 2017}
\date{Syllabus updated: \today} % Activate to display a given date or no date (if empty),
         % otherwise the current date is printed 
         
\tolerance=1
\emergencystretch=\maxdimen
\hyphenpenalty=10000
\hbadness=10000

\begin{document}
\maketitle

\subsection*{\textbf{Course Instructor}}
Dr. Oliver Bonham-Carter\\
\noindent Classroom: Alden Hall 101 \\
\noindent Office Location: Alden Hall 104 \\
\noindent Office Phone: +1 814-332-2880 \\
\noindent Email: \url{obonhamcarter@allegheny.edu} \\
\noindent Web Site: \url{http://www.cs.allegheny.edu/sites/obonhamcarter/} \\
%\noindent Slack Team: \url{cs380Fall2016.slack.com}

\subsection*{\textbf{Instructor's Office Hours}}

\begin{itemize}
	\itemsep 0em
	\item Monday, Wednesday \emph{and} Friday: 11:00 pm -- 12:00 pm (10 minute time slots)
	\item Tuesday: 2:30pm -- 4pm (15 minute time slots)
	\item Thursday: 1:30pm -- 3:00pm (15 minute time slots)
\end{itemize}

\noindent
To schedule a meeting with me during my office hours, please visit my Web site and click the ``Schedule'' link in the top right-hand corner. Now, you can view my calendar or by clicking ``schedule an appointment'' link browse my office hours and schedule an appointment by clicking the correct link to reserve an open time slot. 


\subsection*{\textbf{Course Meeting Schedule}}

Lecture, Discussion, Presentations and Group Work:\\
\indent
Tuesday, Thursday, 11:00 am -- 12:15 pm \\
Laboratory Session:\\
\indent Wednesday, 2:30 pm -- 4:20 pm \\
%Final Examination: Friday, December 12, 2014 at 9:00 am


\subsection*{\textbf{Academic Bulletin Description}}

%\begin{center}

An introduction to the development and application of methods, from the computational and information sciences, for the investigation of biological phenomena. In this interdisciplinary course, students integrate computational techniques with biological knowledge to develop and use analytical tools for extracting, organizing, and interpreting information from genetic sequence data. Often participating in team-based and hands-on activities, students implement and apply useful bioinformatics algorithms. During a weekly laboratory session students
employ cutting-edge software tools and programming environments to complete projects, reporting on their results through both written assignments and oral presentations. Prerequisites: BIO 221 and FSBIO 201, or CMPSC 111. Distribution Requirements: QR, SP.
%\end{center}

%%%%%%
\omitit{
\subsection*{\textbf{The Course}}
\begin{itemize}
\item  An introduction to the development and application of computational approaches to answer biological questions. Students use state-of-the-art bioinformatics software to gain insights into the functionality of the information contained within genomes as well as learn the algorithms behind such applications. Topics include data management, analysis of large-scale biological datasets, genome annotation, and genetics of disease. Unique challenges in the field and the wide range of existing solutions are examined. Prerequisites: BIO 221 and FSBIO 201, or CMPSC 111.  
\item  The course follows two parallel tracks. In the lectures we will learn basic bioinformatics fundamentals and applications, while in the laboratory sessions you will have a larger hands on experience with problem solving and writing programs. The laboratory sessions will be usually tied to the lectures.
\end{itemize}
}%%% end of omitit{}


\subsection*{\textbf{Course Objectives}}

Students successfully completing this class will have developed:
\begin{enumerate}
  \item A “big-picture” view of bioinformatics.
  \item An understanding of the objectives and limitations of bioinformatics.
  \item An understanding of the biological foundations of bioinformatics (genes and genomes, gene expression, etc.).
  \item An understanding of the computational foundations of bioinformatics (programming, databases, etc.).
  \item An understanding of how genetic information is obtained and processed.
  \item The ability to use basic bioinformatics software tools to study genetic information.
\end{enumerate}

\noindent Throughout the semester students also will enhance their ability to write and present ideas about bioinformatics in a clear and compelling fashion. Students will gain practical experience in the design, implementation, and analysis of bioinformatics research during laboratory sessions and a final project. Finally, students will develop a richer understanding of the fascinating connections between biological systems, analysis and automation.


\subsection*{\textbf{Required Textbooks}}
\begin{itemize}

\item \emph{Exploring Bioinformatics: A Project-based Approach, second edition}, by Caroline St. Clair and
Jonathan E. Visick.

\item \emph{Think Python, first edition}, by Allen B. Downey.
	\begin{itemize}
		\item \url{http://greenteapress.com/wp/think-python/}
	\end{itemize}

\end{itemize}


\subsection*{\textbf{Internet Resources}}
\begin{itemize}
  \item Course web page: 
  \begin{itemize}
    \item \url{http://www.cs.allegheny.edu/sites/obonhamcarter/cs300.html}
    \item You can access course materials on the course web page.
  \end{itemize}
  
  \newpage
  
  \item Sakai page: 
  \begin{itemize}
    \item \url{https://sakai.allegheny.edu/}
    \item The course page on Sakai will only be used for reporting student grades.
  \end{itemize}
  
  \item Bitbucket: 
  \begin{itemize}
    \item \url{https://bitbucket.org/}
    \item Bitbucket, a cloud based system, will be used by the instructor for sharing course materials and by the students for submitting assignments.
  \end{itemize}
\end{itemize}



%%%%%%%%%%%
%%%%%%%%%%%


\omitit{
\subsection*{\textbf{Students who want to improve their technical writing skills may consult the following books.}}


\begin{itemize}

\item {\em BUGS in Writing: A Guide to Debugging Your Prose}. Lyn Dupr\'e. Second Edition,  ISBN-10: 020137921X, ISBN-13: 978-0201379211, 704 pages, 1998.

\item {\em Writing for Computer Science}.  Justin Zobel. Second Edition,  ISBN-10: 1852338024, ISBN-13:978-1852338022, 270 pages, 2004.

\item Along with reading the required books, you will be asked to study many additional articles from a wide variety of conference proceedings, journals, and the popular press.
\end{itemize}
}
%%%%%%%%%%%
%%%%%%%%%%%


\subsection*{\textbf{Class Policies}}

\subsubsection*{\textbf{Grading}}

The grade that a student receives in this class will be based on the following categories. All percentages are approximate and, if the need to do so presents itself, it is possible for the assigned percentages to change during the academic semester. 
\color{red}
\begin{center}
  \begin{tabular}{l|l}
\hline
    Class Participation & 15\% \\  %and Instructor Meetings 
    Exams  (Three) & 20\% \\
%    Final Examination & 20\% \\
    Laboratory  Assignments & 30\% \\
    Final Project & 35\% \\
\hline
  \end{tabular}
\end{center}
\color{black}
\noindent
These grading categories have the following definitions:
\vspace*{-.05in}


\begin{itemize}

  \item {\em Class Participation}: All students are required to actively participate during all of the class sessions. Your participation will take forms such as answering questions about the required reading assignments, completing in-class exercises, asking constructive questions of the other members of the class, giving presentations, leading a discussion session in class.% and in the course's Slack channels. 

  \item {\em Exams}: The exams will cover all of the material in their associated module(s). The finalized date for each of the exams will be announced at least one week in advance of the scheduled date. Unless prior arrangements are made with the course instructor, all students will be expected to take these exams on the scheduled date and complete the exams in the stated period of time.
  
  \item {\em Laboratory Assignments}: These assignments invite students to explore the concepts, tools, and techniques associated with the field of bioinformatics.  All of the laboratory assignments require the use of the provided tools to study,  design, implement, and evaluate informatics systems that solve biology problems.  To ensure that students are ready to utilize and develop appropriate software in both other classes at Allegheny College and after graduation, the instructor will assign individuals to teams for some of the laboratory assignments.  Unless specified otherwise, each laboratory assignment will be due at the beginning of the next laboratory session.  Some of the  assignments in this course will expect students to give both a short presentation and a demonstration of the bioinformatics solution that they created.  

  \item {\em Final Project}: This project will present you with an opportunity to design and implement a correct and carefully evaluated bioinformatics solution for a specific problem. Completion of the final project will require you to apply all of the knowledge and skills that you have acquired during the course of the semester to solve a bioinformatics problem. The details for the final project will be given approximately two months before the project due date (during finals week).

  
\end{itemize}


%%%
\omitit{
\subsubsection*{\textbf{Assignment Submission}}

\begin{itemize}
  \item All assignments will have a stated due date and are to be turned in electronically on that due date; all assignments must have headers with your name, date, and the Honor Code pledge of the student(s) completing the work. You must follow proper procedures for submitting your completed programs in order for them to be graded. You will be given instructions on how to do that with your first programming assignment.  For any assignment completed in a group, students will also peer review each group members' contribution to the assignment. 

%The submission of assignments comprises the Honor Code pledge of the student(s) completing the work. For any assignment completed in a group, students must also turn in a one-page reflection that describes each group member's contribution to the submitted deliverables.  

  %All assignments will have a stated due date. \color{red}\textbf{Since solutions guides will be handed out at the beginning of class on due dates, the electronic version of the class assignments are to be turned in at the beginning of the class on that due date. Submissions after the beginning of class are counted as being late.} \color{black} Assignments will be accepted for up to one week past the assigned due date with a 15\% penalty. All late assignments must be submitted at the beginning of the session that is scheduled one week after the due date. 
  \item Late assignments will be accepted for up to one week past the assigned due date with a 15 percent penalty. All of the late assignments must be submitted by the beginning of the session that is scheduled one week after the due date. Unless special arrangements are made with the course instructor, no assignments will be accepted after the late deadline. 
\end{itemize}
}% end of omitit{}



\subsubsection*{\textbf{Assignment Submission}}
%\color{red}\textbf{Since solutions guides will be handed out at the beginning of class on due dates, 

All assignments will have a stated due date. \color{red} The electronic version of the class assignments are to be turned in at the beginning of the lab session on that due date. Submissions after the beginning of class are counted as being late. \color{black}  Assignments will be accepted for up to one week past the assigned due date with a 15\% penalty. All late assignments must be submitted at the beginning of the session that is scheduled one week after the due date. 

  
  \subsubsection*{\textbf{Extensions}}
Unless special arrangements are made with the course instructor, no assignments will be accepted after the late deadline. If you are requesting extensions for a lab assignment, then you are to email me with your request and also provide a \emph{valid reason} for your extension. This request must come before the due date of the lab and not on the due date. Requests will not be granted where the reason appears to be insignificant. Extensions are 24 hours of extra time (after the original due date) and are given out at my discretion. The decision to provide you with an extension (or not) will be weighed in light of fairness to your peers who are still able to complete their labs, regardless of their own busy schedules. 


% All assignments will have a stated due date. The printed version of the assignment is to be turned in at the beginning
% of the class on that due date; the printed materials must be dated and signed with the Honor Code pledge of all the
% student(s) in a group.  When the printed version is submitted, the electronic version of the assignment also must be
% made available to the course instructor in a version control repository. Late assignments will be accepted for up to
% one week past the assigned due date with a 15\% penalty. All late assignments must be submitted at the beginning of
% the session that is scheduled one week after the due date. Unless special arrangements are made with the course
% instructor, no assignments will be accepted after the late deadline. In addition to submitting the required
% deliverables for any assignment completed in a group, students must turn in a one-page document that describes each
% group member's contribution to the submitted deliverables.  

\section*{\textbf{Attendance}}
It is mandatory for all students to attend all classes and laboratory sessions. You will receive a grade deduction of one letter grade on the laboratory submission if you did not attend the scheduled laboratory session, unless prior arrangements have been made with the instructor. If you will not be able to attend a session, then please see the instructor at least one week in advance to describe your situation. Students who miss more than five unexcused sessions will have their final grade in the course reduced by one letter grade. Frequent or prolonged absences due to illness should be documented by the student's doctor, the Health Center, the Dean of Students’ Office, or the office of Student Disability Services. If you need to miss class due to a religious observance, please speak to the instructor in advance to make appropriate arrangements.

\subsection*{\textbf{Use of Laboratory Facilities}}

Throughout the semester, we will investigate many different software tools that bioinformaticians use during the design, implementation, and evaluation of solutions. The course instructor and the system administrator in the department of Computer Science have invested a considerable amount of time to ensure that our laboratories support the completion of all of the assignments and projects. To this end, students are required to complete all of the computational laboratory assignments and the final project while using the Computer Science department's laboratory facilities. The course instructor and the systems administrator normally do not assist students in configuring their personal computers.

\section*{\textbf{Class Preparation}}

 
% The study of the computer science discipline is very challenging.  Students in this class will be challenged to learn
% the principles and practice of software development.  During the coming semester even the most diligent student will
% experience times of frustration when they are attempting to understand a challenging concept or complete a difficult
% laboratory assignment.  In many situations some of the material that we examine will initially be confusing : do not
% despair!  Press on and persevere!
% 
 In order to minimize confusion and maximize learning, students must invest time to prepare for the class discussions, lectures, and laboratory sessions. During the class periods, the course instructor will often pose questions that could require group discussion, the use of an online software tool, the creation or modification of a program or data set, or a group presentation.  In order to help students remain organize and effectively prepare for classes, the course instructor will maintain a class schedule with reading assignments and presentation slides.  


\subsection*{\textbf{A Note on extenuating circumstances}}
%\begin{itemize}
% \item 
If you should find yourself in difficult circumstances that significantly interfere with your ability to prepare for this class and to complete assignments, please inform the instructor immediately so that we can work something out together! Do not wait until the last day of class to ask for exceptions to what is stated in this syllabus. In such a situation, you may also find it helpful to contact the campus Counseling Center (332-4368) in 304 Reis Hall, which is open from 8-5 but also has a 24 hour hotline.
%\end{itemize}

\subsection*{\textbf{Special Needs and Disabilities}}
%\begin{itemize}
% \item 
Students with disabilities who believe they may need accommodations in this class are encouraged to contact Disability Services at (814) 332-2898. Disability Services is part of the Learning Commons and is located in Pelletier Library. Please do this as soon as possible to ensure that approved accommodations are implemented in a timely fashion.
%\end{itemize}


%In order to minimize confusion and maximize learning, students must invest time to prepare for class discussions and lectures.  During the class periods, the course instructor will often pose demanding questions that could require group discussion, the creation of a program or test suite, a vote on a thought-provoking issue, or a group presentation.  Only students who have prepared for class by reading the assigned material and reviewing the current assignments will be able to effectively participate in these discussions.  More importantly, only prepared students will be able to acquire the knowledge and skills that are needed to be successful in both this course and the field of data management.  In order to help students remain organized and effectively prepare for classes, the course instructor will maintain a class schedule with reading assignments and presentation slides. During the class sessions students will also be required to download, use, and modify programs, diagrams, and data sets that are made available through the course Bitbucket repository.


\subsection*{\textbf{Email}}

Using your Allegheny College email address, the instructor will sometimes send out class announcements about matters such as assignment clarifications or changes in the schedule. It is your responsibility to check your email at least once a day and to ensure that you can reliably send and receive emails. This class policy is based on the following statement in {\em The Compass}, the college's student handbook.

\vspace*{-.1in}
\begin{quote}
  ``The use of email is a primary method of communication on campus. \ldots  
All students are provided with a campus email account and address while
  enrolled at Allegheny and are expected to check the account on a regular
  basis.'' 
\end{quote}
\vspace*{-.1in}

\subsubsection*{\textbf{Disability Services}}

The Americans with Disabilities Act (ADA) is a federal anti-discrimination statute that provides comprehensive civil rights protection for persons with disabilities.  Among other things, this legislation requires all students with disabilities be guaranteed a learning environment that provides for reasonable accommodation of their disabilities. Students with disabilities who believe they may need accommodations in this class are encouraged to contact Disability Services at 332-2898.  Disability Services is part of the Learning Commons and is located in Pelletier Library. Please do this as soon as possible to ensure that approved accommodations are implemented in a timely fashion.

\subsubsection*{\textbf{Honor Code}}

The Academic Honor Program that governs the entire academic program at Allegheny College is described in the Allegheny Course Catalogue.  The Honor Program applies to all work that is submitted for academic credit or to meet non-credit requirements for graduation at Allegheny College.  This includes all work assigned for this class (e.g., examinations, laboratory assignments, and the final project).  All students who have enrolled in the College will work under the Honor Program. Each student who has matriculated at the College has acknowledged the following pledge:

\vspace*{-.1in}
\begin{quote}
\emph{I hereby recognize and pledge to fulfill my responsibilities, as defined in the Honor Code, and to maintain the integrity of both myself and the College community as a whole.}
\end{quote}
\vspace*{-.1in}

\noindent Additionally, we expect that you will adhere to the following Department Policy:

\begin{center} \textbf{ Department of Computer Science Honor Code Policy } \end{center}
\vspace*{-.1in}
It is recognized that an important part of the learning process in any course, and particularly in computer science, derives from thoughtful discussions with teachers, student assistants, and fellow students. Such dialogue is encouraged. However, it is necessary to distinguish carefully between the student who discusses the principles underlying a problem with others, and the student who produces assignments that are identical to, or merely variations on, someone else's work. It will therefore be understood that all assignments submitted to faculty of the Department of Computer Science are to be the original work of the student submitting the assignment, and should be signed in accordance with the provisions of the Honor Code. Appropriate action will be taken when assignments give evidence that they were derived from the work of others.

\begin{quote}
Note: You are encouraged to periodically review the specifics of the Honor Code as stated in the College Catalogue, The Compass, and elsewhere.
 \end{quote}

\end{document}
